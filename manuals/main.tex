\documentclass[12pt]{article}
\usepackage[utf8]{inputenc}
\usepackage[russian]{babel}
\usepackage[T1]{fontenc}
\usepackage{graphicx}

\title{Инструкция к изменению данных в ПО}
\author{}
\date{}

\begin{document}
\maketitle

\section{Начало работы}
Для начала работы запустите ярлык, который находится рядом с ярлыком для презентации. Он может называться "управление" или другим подобным образом. \newline
При запуске откроется одна из таблиц.
Панель с синими кнопками переключения таблиц расположена сверху. При нажатии на соотвествующую кнопку отображаемые данные поменяются на те, что находятся в этом разделе (таблице). \newline
Рядом с каждой строкой раздела есть кнопки редактирования и удаления.
\begin{figure}[h]
\centering
\includegraphics[width=0.8\textwidth]{1.png}
\caption{Начальная страница}
\end{figure}


\section{Добавление данных}
Чтобы добавить новую строку в существующий раздел, нажмите на зеленую кнопку в левом верхнем углу экрана "Добавить запись". Перед вами откроется панель, где необходимо внести все данные в соотвествии с форматом хранимых данных. После этого нажать на кнопку "Добавить запись". При успешном добавлении внизу появится зеленое уведомление.
\begin{figure}[h]
\centering
\includegraphics[width=0.8\textwidth]{2.png}
\caption{Панель добавления данных}
\end{figure}


\section{Удаление данных}
Чтобы удалить строку раздела нажмите на значок корзины в правом углу нужной строки. У вас запросят подтверждение: нажмите "Удалить" чтобы удалить строку. 
\begin{figure}[h]
\centering
\includegraphics[width=0.8\textwidth]{3.png}
\caption{Панель удаления данных}
\end{figure}

\section{Редактирование данных}
Для редактирования существующей строки нажмите на значок редактирования в правом углу нужной строки. Перед вами откроется панель, где можно изменить текстовые данные в соотвествии с форматом хранимых данных. \newline 
Каждое поле в открывшейся панели является измениемым, то есть в него можно что-то вписывать или удалять лишнее. Если текст не был никак изменен, кнопка редактирования снизу экрана останется неактивной. Если изменения были совершены, то кнопка станет активной (зеленого цвета). Для применения изменений нажмите на эту кнопку. При успешном изменении внизу появится зеленое уведомление.
\begin{figure}[h]
\centering
\includegraphics[width=0.8\textwidth]{4.png}
\caption{Неактивная панель редактирование данных}
\end{figure}
\begin{figure}[h]
\centering
\includegraphics[width=0.8\textwidth]{5.png}
\caption{Активная панель редактирование данных}
\end{figure}

\end{document}